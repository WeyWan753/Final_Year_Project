
\chapter{SU(2) as a double cover of SO(3)}

In this report, we aim to show that SU(2) is a double cover of SO(3). Specifically we will construct an explcit group homomorphism from SU(2) to SO(3) and show that it is surjective with kernel \{$\pm I$\}, therefore establishing that the map is 2-to-1.

\section{Pauli matrices}

The pauli matrices $\sigma_1, \sigma_2, \sigma_3$ are defined by the $2 \times 2$ matrices : 
\[
	\sigma_1 = \begin{pmatrix}
	0 & 1 \\
	1 & 0 \\
	\end{pmatrix}
	\quad
	\sigma_2 = \begin{pmatrix}
	0 & -i \\
	i & 0 \\
	\end{pmatrix}
	\quad
	\sigma_3 = \begin{pmatrix}
	1 & 0 \\
	0 & -1 \\
	\end{pmatrix}
	\quad
\]

$\newline$
The Pauli matrices has the following properties :
\[
	\begin{aligned}
	\sigma_i^2 &= I, \quad \text{for } i = 1,2,3 \\
	\sigma_i \sigma_j + \sigma_j \sigma_i &= 0, \quad \text{for } i \neq j
	\end{aligned}
\]
$\newline$



Claim. The Paulie matrices together with the identity $I$ form a basis of $M_2(\C)$ over $\C$.
\begin{proof}
A generic matrix $M \in M_2(\C)$ has the form of
\[
	M = \begin{pmatrix}
		a & b \\
		c & d \\
		\end{pmatrix}
		,\quad \quad
		a,b,c,d \in \C
\]
We will show that there exist complex numbers $\alpha, \beta, \gamma, \delta$ such that 
\[
	M = \alpha I + \beta \sigma_1 + \gamma \sigma_2 + \delta \sigma_3
\]
Compute the right-hand side:
\[
	\alpha I + \beta \sigma_1 + \gamma \sigma_2 + \delta \sigma_3 = \begin{pmatrix}
	\alpha + \delta & \beta - i \gamma \\
	\beta + i \gamma & \alpha - \delta \\
	\end{pmatrix}
\]
Equating entries with $M$ gives the linear system
\[
\begin{cases}
	\alpha + \delta = a \\
	\alpha - \delta = d \\
	\beta - i \gamma = b \\
	\beta + i \gamma = c \\
\end{cases}
\]
From the first two equations,
\[
	\alpha = \frac{a + d}{2}, \quad \quad \delta = \frac{a - d}{2}
\]
From the last two,
\[
	\beta = \frac{b + c}{2}, \quad \quad \gamma = \frac{c - b}{2i}
\]

These formulae give a solution for $\alpha,\beta,\gamma,\delta \in \C$. Hence the four matrices span $M_2(\C)$. If we now consider the zero matrix, 
\[	
	M = \begin{pmatrix}
		0 & 0 \\
		0 & 0 \\
		\end{pmatrix}
\]
Then the same decomposition
\[	\alpha I + \beta \sigma_1 + \gamma \sigma_2 + \delta \sigma_3 = M	\]
forces $\alpha = \beta = \gamma = \delta = 0$.
$\newline\newline$
This shows that the set $\{I, \sigma_1, \sigma_2, \sigma_3 \}$ is linearly independent. Therefore the Pauli matrices together with the identity form a basis of $M_2(\C)$ over $\C$

\end{proof}
We recall that the definition of the traces of a matrix and some of its properties. 
For a square matrix $A = (a_{ij}) \in M_n(\C)$, the trace of $A$ is defined by 
\[
	\text{Tr}(A) = \sum_{i = 1}^n a_{ii}
\]
That is the trace is the sum of the diagonal entries of $A$ 
$\newline \newline$
Properties of the trace. \\
For any $A,B \in M_n(\C)$ and any scalar $c \in \C$, the trace satisfies :
\begin{enumerate}
	\item Linearity: \\
	\[	\text{Tr}(A + B) = \text{Tr}(A) + \text{Tr}(B),\quad \text{and} \quad \text{Tr}(cA) = c\, \text{Tr}(A) \] \\
	\item Cyclic property: \\
	\[	\text{Tr}(AB) = \text{Tr}(BA)	\]
	\begin{proof}
		let $A = (a_{ij})$ and $B = (b_{ij})$. Then
		\[
			\text{Tr}(A) = \sum_{i = 1}^n (AB)_{ii} = \sum_{i=1}^n \sum_{j=1}^n a_{ij}b_{ji} = \sum_{j=1}^n \sum_{i=1}^n b_{ji}a_{ij} = \sum_j^n (BA)_{jj} = \text{Tr}(BA)
		\]
	\end{proof}
	\item Transpose invariance: \\
	\[	\text{Tr}(A^T) = \text{Tr}(A)	\]
	\item Conjugate transpose: \\
	\[	\text{Tr}(A^*) = \overbar{\text{Tr}(A)}	\]
	\item Trace of identity:
	\[	\text{Tr}(I_n) = n	\]
\end{enumerate}
The proofs of the remaining properties are straightforward and follow directly from the definitions, so we omit them here.




We now consider the real subspace $V$ of $M_2(\C)$ defined by
\[
	V = \text{span}_\R \{\sigma_1, \sigma_2, \sigma_3 \}
\]
That is,
\[
	V = \{x_1\sigma_1 + x_2\sigma_2 + x_3\sigma_3 \text{ } | \text{ } x_1,x_2,x_3 \in \R \}
\]

Every element $A \in V$ can be expressed uniquely as 
\[
	A = x_1\sigma_1 + x_2\sigma_2 + x_3\sigma_3 = 
	\begin{pmatrix}
	x_3 & x_1 - ix_2 \\
	x_1 + ix_2 & -x_3 \\
	\end{pmatrix}
\]
\begin{proposition}
	The space $V$ can equivalently be described as the subset of $M_2(\C)$ consisting of all Hermitian and traceless matrices, That is,
\[	V = \{	A \in M_2(\C) \, | \, A^* = A, \, \text{Tr}(A) = 0	\}	\]Where a matrix is called Hermitian if it is equal to its conjugate transpose, that is, $A^* = A$.

\end{proposition}
We can consider $A$ as a map 
\[
	\begin{aligned}
	A : \text{ } &\R^3 \rightarrow V, \\
		&\vec{x} \mapsto x_1\sigma_1 + x_2\sigma_2 + x_3\sigma_3
	\end{aligned}
\]

\begin{proposition}
	The map $A : \R^3 \rightarrow V$ is a real vector space isomorphism.
\end{proposition}

\begin{proof}
$ $\newline
\begin{enumerate} 
	\item Linearity: 
	\newline
	For any $\vec{x} = (x_1,x_2,x_3), \vec{y} = (y_1,y_2,y_3) \in \R^3$ and $c \in \R$,
\[
	\begin{aligned}
	A(\vec{x}+\vec{y}) &= A(x_1 + y_1, x_2 + y_2, x_3 + y_3) \\
					   &= (x_1 + y_1) \sigma_1 + (x_2 + y_2) \sigma_2 + (x_3 + y_3) \sigma_3 \\
					   &= x_1 \sigma_1 + y_1 \sigma_2 + x_2 \sigma_3 + y_2 \sigma_1 + x_3 \sigma_2 + y_3 \sigma_3 \\
					   &= x_1 \sigma_1 + x_2 \sigma_2 + x_3 \sigma_3 + y_1 \sigma_1 + y_2 \sigma_2 + y_3 \sigma_3 \\
					   &= A(\vec{x}) + A(\vec{y}) \\ \\
	A(c\vec{x}) &= A(cx_1, cx_2, cx_3) \\
				&= cx_1\sigma_1 + cx_2\sigma_2 + cx_3\sigma_3 \\
				&= c(x_1\sigma_1 + x_2\sigma_2 + x_3\sigma_3) \\
	\end{aligned}
\]
	\item Injectivity:
	\newline
	Suppose $A(\vec{x}) = A(\vec{y})$. Then
\[	
x_1\sigma_1 + x_2\sigma_2 + x_3\sigma_3 = y_1\sigma_1 + y_2\sigma_2 + y_3\sigma_3 \]
Moving all terms to the left hand side gives
\[
(x_1 - y_1) \sigma_1 + (x_2 - y_2)\sigma_2 + (x_3 - y_3)\sigma_3 = 0 \\
\]
Since $\sigma_1,\sigma_2,\sigma_3$ are linearly independent, each coefficient must be zero. Therefore, 
\[	x_1 = y_1, \quad x_2 = y_2, \quad x_3 = y_3.	\]
Thus $A$ is injective.
	\item Surjectivity:
	\newline
	By definition, every element of $V$ is a real linear combination of $\sigma_1, \sigma_2, \sigma_3$, hence for any $B \in V$, there exist $\vec{x} \in \R^3$ with $B = A(\vec{x})$. Thus $A$ is surjective.
\end{enumerate}
Since $A$ is linear, injective and surjective, it is a bijective linear map. Hence an isomorphism of real vector spaces.
\end{proof}

We now compute the trace inner product of the image under $A$. \\
For any $\vec{x} = (x_1,x_2,x_3), \vec{y} = (y_1,y_2,y_3) \in R^3$, we have 
\[	A(\vec{x})A(\vec{y}) = (x_1\sigma_1 + x_2\sigma_2 + x_3\sigma_3)(y_1\sigma_1 + y_2\sigma_2 + y_3\sigma_3) = \sum_{1 \leq i,j \leq 3} x_iy_j\sigma_i\sigma_j	\]
Taking the trace yield
\[	\text{Tr}(A(\vec{x})A(\vec{y})) = \sum_{1 \leq i,j \leq 3} x_iy_j\text{Tr}(\sigma_i\sigma_j)	\]
Using the relation
	\[	\text{Tr}(AB) = \text{Tr}(BA) \quad \text{and} \quad \sigma_i\sigma_j + \sigma_j\sigma_i = 0 \quad \text{for  } i \neq j	\]
Since $\sigma_i^2 = I$ it follows that,
\[	\text{Tr}(A(\vec{x})A(\vec{y})) = 	2\sum_{i = 1}^3 x_iy_i = 2\,\vec{x}\cdot\vec{y}\]
Therefore we can define an inner product in $V$ that is equivalent to the one in $\R^3$ by
\[	A(\vec{x}) \cdot A(\vec{y})	= \frac{1}{2}\text{Tr}(A(\vec{x})A(\vec{y}))\]

Now let us define a transformation $T_U$ on $V$ where $U \in$ SU(2) by
\[	T_U(A(\vec{x})) = UA(\vec{x})U^{-1} = UA(\vec{x})U^*.	\]
The map $T_U$ indeed yields an element of $V$, as shown below.
We regard $V$ as the subspace of $M_2(\C)$ consisting of all Hermitian, traceless matrices. These two properties completely charaterize $V$,as established earlier. Now observe that
\[
	\begin{cases}
		(T_U(A(\vec{x})))^* = (UA(\vec{x})U^*)^* = UA(\vec{x})U^* = T_U(A(\vec{x})), \\
		\text{Tr}(T_U(A(\vec{x})) = \text{Tr}(UA(\vec{x})U^{-1}) = \text{Tr}(U^{-1}UA(\vec{x})) = \text{Tr}(A(\vec{x})) = 0.
	\end{cases}
\]
so $T_U(A(\vec{x})) \in V$. We can consider the inner product of $T_U(A(\vec{x}))$ and $T_U(A(\vec{y}))$,

\[
	\begin{aligned}
	T_U(A(\vec{x})) \cdot T_U(A(\vec{y})) &= \frac{1}{2}\text{Tr}(UA(\vec{x})U^{-1}UA(\vec{y})U^{-1}) \\
	&= \frac{1}{2}\text{Tr}(UA(\vec{x})A(\vec{y})U^{-1})\\
	&= \frac{1}{2}\text{Tr}(U^{-1}UA(\vec{x})A(\vec{y}))\\
	&= \frac{1}{2}\text{Tr}(A(\vec{x})A(\vec{y}))\\
	&= A(\vec{x}) \cdot A(\vec{y}) \\
	\end{aligned}
\]
This shows that $T_U$ preserve the inner product on $V$. Now, consider the corresponding transformation on $\R^3$ instead of $V$. Define
\[	\phi_U = A^{-1} \circ T_U \circ A,	\]
which represents $T_U$ as a transformation action on $\R^3$. From the previous equality, we have
\[
\begin{aligned}
\phi_U(\vec{x}) \cdot \phi_U(\vec{y})
&= \frac{1}{2}\text{Tr}( A(\phi_U(\vec{x}))\, A(\phi_U(\vec{y})) ) \\
&= \frac{1}{2}\text{Tr}( A(A^{-1}(T_U(A(\vec{x}))))\, A(A^{-1}(T_U(A(\vec{y})))) ) \\
&= \frac{1}{2}\text{Tr}( T_U(A(\vec{x}))\, T_U(A(\vec{y})) ) \\
&= \frac{1}{2}\text{Tr}( A(\vec{x})\, A(\vec{y}) ) \\
&= \vec{x} \cdot \vec{y}\, 
\end{aligned}
\]
for all $\vec{x},\vec{y} \in \mathbb{R}^3$.
Hence $\phi_U$ preserves the inner product on $\mathbb{R}^3$.
Any linear map on $\mathbb{R}^3$ that preserves the inner product is orthogonal,
and therefore
\[
    \phi_U \in O(3).
\]


Accordingly, we introduce the map
\[
\phi : SU(2) \to O(3), \qquad U \mapsto \phi_U,
\]
where
\[
\phi_U = A^{-1} \circ T_U \circ A .
\]

\begin{proposition}
The map $\phi : SU(2) \to SO(3)$ is continuous.  
We establish this by first writing an explicit expression for $\phi(U)$, and then we verify that each coordinate function is continous. 
\end{proposition}

\begin{proof}
We regard $SU(2)$ as a subspace of $M_2(\mathbb{C})$, which is identified with $\mathbb{R}^8$ with the usual topology. Likewise, $O(3)$ is a subspace of $M_3(\mathbb{R}) \cong \mathbb{R}^9$. A function into $M_3(\mathbb{R})$ is continuous if and only if each of its nine matrix-entry coordinate functions is continuous.
\\
\\
Let $e_1,e_2,e_3$ be the standard basis of $\mathbb{R}^3$. The map $A : \mathbb{R}^3 \to V \subset M_2(\mathbb{C})$ satisfies
\[
x \cdot y = \frac12 \mathrm{Tr}(A(x)A(y)).
\]


For $i,j \in \{1,2,3\}$, the $(i,j)$\,-entry of the matrix $\phi(U)$ is
\[
(\phi(U))_{ij}
= e_i \cdot \phi_U(e_j)
= \frac12 \mathrm{Tr}\!\left( A(e_i)\,A(\phi_U(e_j)) \right).
\]
Since
\[
A(\phi_U(e_j)) = U\,A(e_j)\,U^\ast,
\]
we obtain
\[
(\phi(U))_{ij}
= \frac12 \mathrm{Tr}\!\left( A(e_i)\,U\,A(e_j)\,U^\ast \right)
= \frac12 \mathrm{Tr}\!\left( \sigma_i\,U\,\sigma_j\,U^\ast \right).
\]

The expression on the right side is built from the following operations:
matrix multiplication, the adjoint map $U \mapsto U^\ast$, and the trace. 
All of these operations are continuous in the matrix entries of $U$.
Therefore each coordinate function $(\phi(U))_{ij}$ is continuous on $SU(2)$.

Since all nine coordinate functions of $\phi$ are continuous, the map
\[
\phi : SU(2) \longrightarrow M_3(\mathbb{R})
\]
is continuous. Because $O(3)$ has the subspace topology from $M_3(\mathbb{R})$, the restricted map
\[
\phi : SU(2) \to O(3)
\]
is continuous as well.
\end{proof}
We now return to the fact that $\varphi_{U} \in O(3)$. This does not imply that every
element of $O(3)$ is obtained from some $U \in SU(2)$, so it is more accurate to
regard the codomain of $\varphi$ as its image $\operatorname{Ran}(\varphi)$.

Earlier in this report, we have already shown that every
\[
U \in SU(2)
\]
can be written as
\[
U =
\begin{pmatrix}
\alpha & \beta \\
-\overline{\beta} & \overline{\alpha}
\end{pmatrix},
\qquad \alpha, \beta \in \mathbb{C}, \qquad |\alpha|^{2} + |\beta|^{2} = 1.
\]
Writing
\[
\alpha = a_{1} + i a_{2}, \qquad \beta = b_{1} + i b_{2},
\]
with $a_{1}, a_{2}, b_{1}, b_{2} \in \mathbb{R}$, the condition
\[
|\alpha|^{2} + |\beta|^{2} = 1
\]
becomes
\[
a_{1}^{2} + a_{2}^{2} + b_{1}^{2} + b_{2}^{2} = 1.
\]
This is precisely the equation of the $3$-sphere $S^{3} \subset \mathbb{R}^{4}$.
Therefore, under this identification, $SU(2)$ is homeomorphic to $S^{3}$.
Since $S^{3}$ is a connected topological space, it follows that $SU(2)$ is connected as well.





\begin{proposition}
Let $f : X \to Y$ be a continuous function between topological spaces.
If $X$ is connected, then the image $\operatorname{Ran}(f)$ is connected.
\end{proposition}

\begin{proof}
Assume for contradiction that $\operatorname{Ran}(f)$ is not connected.
Then there exist nonempty open sets $A, B \subseteq \operatorname{Ran}(f)$ such that
\[
A \cup B = \operatorname{Ran}(f), \qquad A \cap B = \varnothing.
\]

Consider the preimages $f^{-1}(A)$ and $f^{-1}(B)$. We make the following observations.

\medskip
\noindent
\textbf{1. $f^{-1}(A)$ and $f^{-1}(B)$ are nonempty.}  
Since $A$ and $B$ are nonempty subsets of $\operatorname{Ran}(f)$, there exist
$a, b \in X$ such that $f(a) \in A$ and $f(b) \in B$.  
Thus $a \in f^{-1}(A)$ and $b \in f^{-1}(B)$.

\medskip
\noindent
\textbf{2. $f^{-1}(A)$ and $f^{-1}(B)$ are open in $X$.}  
Because $f$ is continuous and $A$ and $B$ are open in $\operatorname{Ran}(f)$
(with the subspace topology), their preimages are open in $X$.

\medskip
\noindent
\textbf{3. $f^{-1}(A)$ and $f^{-1}(B)$ are disjoint.}  
If $x \in f^{-1}(A) \cap f^{-1}(B)$, then
\[
f(x) \in A \quad \text{and} \quad f(x) \in B,
\]
which contradicts $A \cap B = \varnothing$.  
Hence
\[
f^{-1}(A) \cap f^{-1}(B) = \varnothing.
\]

\medskip
\noindent
\textbf{4. Their union is all of $X$.}  
Let $x \in X$. Since $f(x) \in \operatorname{Ran}(f) = A \cup B$, we have
$f(x) \in A$ or $f(x) \in B$.  
Thus
\[
x \in f^{-1}(A) \cup f^{-1}(B),
\]
so
\[
f^{-1}(A) \cup f^{-1}(B) = X.
\]

\medskip
Together, these four properties show that $f^{-1}(A)$ and $f^{-1}(B)$ form
a separation of $X$, contradicting the assumption that $X$ is connected.
Therefore $\operatorname{Ran}(f)$ must be connected.
\end{proof}

\begin{proposition}
The subsets $\operatorname{SO}(3)$ and $-\operatorname{SO}(3)=\{A\in O(3):\text{det}(A)=-1\}$ form a separation of $O(3)$.
In particular, both $\operatorname{SO}(3)$ and $-\operatorname{SO}(3)$ are nonempty, disjoint, open (in the subspace topology of $O(3)$), and
\[
\operatorname{SO}(3)\cup\big(-\operatorname{SO}(3)\big)=O(3).
\]
Hence $O(3)$ is disconnected and has exactly these two connected components.
\end{proposition}

\begin{proof}
We verify the required properties in turn.

\medskip
\noindent\textbf{Nonemptiness.} \quad The identity matrix $I\in\operatorname{SO}(3)$, so $\operatorname{SO}(3)\neq\varnothing$.  
Also $-I\in O(3)$ and $\text{det}(-I)=-1$, so $-\operatorname{SO}(3)\neq\varnothing$.

\medskip
\noindent\textbf{Disjointness.} \quad If $A\in\operatorname{SO}(3)\cap(-\operatorname{SO}(3))$, then $\text{det}(A)=1$ and $\text{det}(A)=-1$ simultaneously, which is impossible. Hence
\[
\operatorname{SO}(3)\cap\big(-\operatorname{SO}(3)\big)=\varnothing.
\]

\medskip
\noindent\textbf{Union.} \quad For any $A\in O(3)$ one has $\text{det}(A)=\pm 1$. Thus every $A\in O(3)$ lies in $\operatorname{SO}(3)$ or in $-\operatorname{SO}(3)$, so
\[
\operatorname{SO}(3)\cup\big(-\operatorname{SO}(3)\big)=O(3).
\]

\medskip
\noindent\textbf{Openness.} \quad We use the fact that the determinant map
\[
\text{det} : M_{3}(\mathbb{R}) \longrightarrow \mathbb{R}
\]
is continuous. The values of $\text{det}$ on $O(3)$ are exactly $\{1,-1\}$. Choose disjoint open intervals $U_{1},U_{-1}\subset\mathbb{R}$ with $1\in U_{1}$ and $-1\in U_{-1}$ (for example $U_{1}=(\tfrac{1}{2},\tfrac{3}{2})$ and $U_{-1}=(-\tfrac{3}{2},-\tfrac{1}{2})$).

By continuity of $\text{det}$, the preimages $\text{det}^{-1}(U_{1})$ and $\text{det}^{-1}(U_{-1})$ are open in $M_{3}(\mathbb{R})$. Since $\text{det}(O(3))=\{1,-1\}$ and the intervals $U_1,U_{-1}$ are chosen to separate $1$ and $-1$, we have
\[
\operatorname{SO}(3)=\text{det}^{-1}(\{1\})\cap O(3)=\big(\text{det}^{-1}(U_{1})\cap O(3)\big),
\]
and similarly
\[
-\operatorname{SO}(3)=\text{det}^{-1}(\{-1\})\cap O(3)=\big(\text{det}^{-1}(U_{-1})\cap O(3)\big).
\]
Hence both $\operatorname{SO}(3)$ and $-\operatorname{SO}(3)$ are open in the subspace topology of $O(3)$.

\medskip
Since $\operatorname{SO}(3)$ and $-\operatorname{SO}(3)$ are nonempty, disjoint, open in $O(3)$ and their union equals $O(3)$, they form a separation of $O(3)$. Therefore $O(3)$ is disconnected and its two connected components are precisely $\operatorname{SO}(3)$ and $-\operatorname{SO}(3)$.
\end{proof}


\begin{proposition}
Let $X$ be a topological space with a separation $A$ and $B$.
If $Y\subseteq X$ is connected, then $Y$ is contained entirely in $A$ or entirely in $B$.
\end{proposition}

\begin{proof}
Suppose for contradiction that $Y$ intersects both $A$ and $B$. 
Then there exist points $y_A\in Y\cap A$ and $y_B\in Y\cap B$.

Define
\[
A_Y = Y\cap A, \qquad B_Y = Y\cap B.
\]

Both sets are nonempty. Since $A$ and $B$ are open in $X$, 
the intersections $A_Y$ and $B_Y$ are open in the subspace $Y$. 
They are disjoint because $A$ and $B$ are disjoint, and their union is
\[
A_Y\cup B_Y = (Y\cap A)\cup (Y\cap B) = Y\cap (A\cup B) = Y.
\]

Thus $A_Y$ and $B_Y$ form a separation of $Y$, contradicting the fact that $Y$ is connected.
Therefore $Y$ must lie entirely in $A$ or entirely in $B$.
\end{proof}



\begin{proposition}
Let $\phi:SU(2)\to O(3)$ be the continuous map constructed above.
Then $\operatorname{Ran}(\phi)\subseteq\operatorname{SO}(3)$.
\end{proposition}

\begin{proof}
Since $\phi$ is continuous and $SU(2)$ is connected (homeomorphic to $S^3$),
the image $\operatorname{Ran}(\phi)=\phi(SU(2))$ is a connected subset of $O(3)$.

Recall that $O(3)=\operatorname{SO}(3)\cup\big(-\operatorname{SO}(3)\big)$ and that
$\operatorname{SO}(3)$ and $-\operatorname{SO}(3)=\{A\in O(3):\text{det}(A)=-1\}$
form a separation of $O(3)$ (they are nonempty, disjoint, open in the subspace topology,
and their union is $O(3)$). By the preceding proposition on connected subspaces, any connected
subset of $O(3)$ must lie entirely in one of the two separated pieces. Hence
\[
\operatorname{Ran}(\phi)\subseteq \operatorname{SO}(3)
\quad\text{or}\quad
\operatorname{Ran}(\phi)\subseteq -\operatorname{SO}(3).
\]

Finally, note that $\phi(I)=I$, because
$T_{I}(A(x))=IA(x)I^{-1}=A(x)$ and thus $\phi_{I}=\mathrm{id}_{\mathbb{R}^3}$.
In particular $I\in\operatorname{Ran}(\phi)$ and $I\in\operatorname{SO}(3)$, so
$\operatorname{Ran}(\phi)$ cannot be contained in $-\operatorname{SO}(3)$. Therefore
\[
\operatorname{Ran}(\phi)\subseteq\operatorname{SO}(3),
\]
as required.
\end{proof}









\section{Open neighborhoods generate connected topological groups, and application to \(\phi:SU(2)\to SO(3)\)}

\begin{definition}[Topological group]
A \emph{topological group} is a group \(G\) equipped with a topology such that
the multiplication map \(G\times G\to G,\ (x,y)\mapsto xy\) and the inversion map
\(G\to G,\ x\mapsto x^{-1}\) are continuous.
\end{definition}

\begin{lemma}
Let \((X,\tau)\) be a topological space. If \(X\) is connected then the only subsets
of \(X\) which are both open and closed (clopen) are \(\varnothing\) and \(X\) itself.
\end{lemma}

\begin{proof}
Let \(C\subset X\) be clopen. If \(C\neq\varnothing\) and \(C\neq X\) then
\(C\) and \(X\setminus C\) are nonempty disjoint open sets whose union is \(X\),
contradicting connectedness. Hence any clopen subset must be either \(\varnothing\)
or \(X\).
\end{proof}

\begin{definition}[Symmetric neighborhood]
Let \(G\) be a topological group with identity \(e\). An open neighborhood
\(V\subset G\) of \(e\) is called \emph{symmetric} if \(V^{-1}=\{v^{-1}:v\in V\}=V\).
\end{definition}

\begin{lemma}
If \(U\) is any open neighborhood of the identity in a topological group \(G\),
then \(V:=U\cap U^{-1}\) is an open symmetric neighborhood of the identity contained
in \(U\).
\end{lemma}

\begin{proof}
The inversion map is continuous, so \(U^{-1}\) is open. The intersection \(V=U\cap U^{-1}\)
is therefore open, contains \(e\), and satisfies \(V^{-1}=(U\cap U^{-1})^{-1}=U^{-1}\cap U=V\).
\end{proof}

\begin{proposition}\label{prop:H-is-clopen-subgroup}
Let \(G\) be a topological group and let \(V\subset G\) be an open symmetric neighborhood
of the identity \(e\). For each integer \(k\ge 1\) define
\[
V^{k}:=\underbrace{V\cdot V\cdots V}_{k\text{ times}},
\qquad
H:=\bigcup_{k\ge 1} V^{k}.
\]
Then \(H\) is an open and closed subgroup of \(G\).
\end{proposition}

\begin{proof}
\emph{(Subgroup).}
If \(x\in V^{m}\) and \(y\in V^{n}\) then \(xy\in V^{m+n}\subset H\), so \(H\) is closed
under products. Since \(V\) is symmetric, \((V^{m})^{-1}=V^{m}\), hence \(x^{-1}\in V^{m}\subset H\).
The identity \(e\in V\subset H\). Associativity is inherited from \(G\). Thus \(H\) is a subgroup.

\medskip\noindent
\emph{(Openness).}
We prove by induction that each \(V^{k}\) is open. The base \(V^{1}=V\) is open by assumption.
Assume \(V^{k}\) is open. Then
\[
V^{k+1}=V\cdot V^{k}=\bigcup_{a\in V} aV^{k}.
\]
For each fixed \(a\in V\) the left-translation \(L(a):G\to G,\ L(a)(x)=ax\), is a homeomorphism
(since multiplication is continuous and \(L(a)^{-1}=L(a^{-1})\) is continuous). Hence \(aV^{k}=L(a)(V^{k})\)
is open. As a union of open sets, \(V^{k+1}\) is open. By induction all \(V^{k}\) are open, and so
\(H=\bigcup_{k\ge1}V^{k}\) is open.

\medskip\noindent
\emph{(Closedness).}
Each left coset \(gH\) is open because \(H\) is open and left translations are homeomorphisms.
Therefore the complement \(G\setminus H=\bigcup_{g\notin H} gH\) is a union of open sets, hence open.
Thus \(H\) is closed as well.

Combining the above, \(H\) is an open-and-closed subgroup of \(G\), as required.
\end{proof}

\begin{corollary}\label{cor:open-neighborhoods-generate-connected-group}
If \(G\) is a connected topological group and \(U\) is any open neighborhood of the
identity \(e\), then
\[
G=\bigcup_{k\ge1} U^{k}.
\]
Equivalently, every open neighborhood of \(e\) topologically generates \(G\).
\end{corollary}

\begin{proof}
By the previous lemma we may replace \(U\) by the symmetric open neighborhood
\(V=U\cap U^{-1}\subset U\). Form \(H=\bigcup_{k\ge1}V^{k}\). By
Proposition \ref{prop:H-is-clopen-subgroup}, \(H\) is a nonempty open-and-closed subgroup
of \(G\). Since \(G\) is connected and \(H\neq\varnothing\), the only possibility is \(H=G\).
Because \(V\subset U\) we have \(\bigcup_{k\ge1}U^{k}\supset\bigcup_{k\ge1}V^{k}=G\),
hence \(\bigcup_{k\ge1}U^{k}=G\).
\end{proof}

\medskip
\noindent\textbf{Application to \(\phi:SU(2)\to SO(3)\).}

Let \(\phi:SU(2)\to SO(3)\) be the continuous group homomorphism constructed earlier.
We state without proof that there exist open neighborhoods
\(A\subset SU(2)\) of the identity and \(B\subset SO(3)\) of the identity such that
the restricted map \(\phi|_{A}:A\to B\) is a homeomorphism. Then:

\begin{itemize}
  \item \(B=\phi(A)\subset\operatorname{Ran}(\phi)\).
  \item Since \(\phi\) is a homomorphism, for every \(k\ge 1\) we have
        \(\phi(A^{k})=\phi(A)^{k}=B^{k}\), so \(B^{k}\subset\operatorname{Ran}(\phi)\).
  \item Therefore \(\bigcup_{k\ge1}B^{k}\subset\operatorname{Ran}(\phi)\).
  \item By Corollary \ref{cor:open-neighborhoods-generate-connected-group} (applied to \(G=SO(3)\),
        which is connected), \(\bigcup_{k\ge1}B^{k}=SO(3)\).
\end{itemize}

Hence \(SO(3)\subset\operatorname{Ran}(\phi)\). Together with the previously established
reverse inclusion \(\operatorname{Ran}(\phi)\subset SO(3)\), we conclude
\[
\operatorname{Ran}(\phi)=SO(3).
\]

\qed

